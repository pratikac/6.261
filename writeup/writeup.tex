\documentclass[letterpaper, 10pt, twocolumn, reqno, fleqn]{amsart}

\usepackage{mathpazo}
\usepackage{amsmath, amsfonts, amssymb, bm, enumerate, url, setspace}
\usepackage[boxruled, vlined, linesnumbered]{algorithm2e}
\usepackage[usenames,dvipsnames,svgnames,table]{xcolor}
\usepackage[pdftex, xetex]{graphicx}
\usepackage[font={small}]{caption}
\usepackage{float, colortbl, tabularx, multirow, subcaption, environ, wrapfig, textcomp}
\usepackage[normalem]{ulem}

\usepackage{pgf, tikz, framed}
%\usetikzlibrary{arrows,positioning,automata,shadows,fit,shapes}

\usepackage[english]{babel}
\usepackage{blindtext}

\usepackage[bookmarks=true]{hyperref}
\usepackage[hyperpageref]{backref}
\hypersetup{
colorlinks=true, linkcolor=blue, citecolor=darkgreen, filecolor=magenta, urlcolor=blue,
}
\usepackage{url, cite}

\newcommand{\pc}[2]{{\color{blue} #1}\marginpar{\tiny\noindent{\raggedright{\color{darkgreen}[PC]}\color{brown}{ #2} \par}}}
\newcommand{\mtodo}[1]{{\color{gray}#1}\marginpar{\tiny\noindent{\raggedright{\color{darkgreen}[MILD TODO]}}}}

%\newcommand{\myclearpage}{\clearpage}
\newcommand{\myclearpage}{}

% math macros
\newcommand{\abs}[1]{\ensuremath \left| #1 \right|}
\newcommand{\norm}[1]{\ensuremath \lVert#1\rVert}
\newcommand{\given}{\, \vert \,}
\renewcommand{\cal}[1]{\ensuremath \mathcal{#1}}
\newcommand{\ag}[1]{\ensuremath \left\langle#1\right\rangle}

% shorcuts
\newcommand{\aeq}[1]{\begin{align} #1 \end{align}}
\newcommand{\aeqs}[1]{\begin{align*} #1 \end{align*}}
\newcommand{\beq}[1]{\begin{equation}#1\end{equation}}
\newcommand{\beqs}[1]{\begin{equation*}#1\end{equation*}}
\newcommand{\trm}[1]{\ensuremath \textrm{#1}}
\newcommand{\enum}[2]{\begin{enumerate}[#1]{#2}\end{enumerate}}
\newcommand{\ilist}[1]{\begin{itemize}{#1}\end{itemize}}
\newcommand{\clist}[1]{\begin{itemize}\setlength{\itemsep}{0pt}
\setlength{\parsep}{0pt}
{#1}\end{itemize}}
\newcommand{\bmat}[1]{\begin{bmatrix}#1\end{bmatrix}}
\newcommand{\mpage}[2]{\begin{center}
\begin{minipage}{#1}#2\end{minipage}\end{center}}
\newcommand{\la}{\leftarrow}
\newcommand{\ra}{\rightarrow}

% theorems
\usepackage{theorem}
\newtheorem{theorem}{Theorem}

\newtheorem{proposition}[theorem]{Proposition}
\newtheorem{lemma}[theorem]{Lemma}
\newtheorem{corollary}[theorem]{Corollary}
\newtheorem{problem}[theorem]{Problem}

\theoremstyle{definition}
\newtheorem{definition}[theorem]{Definition}
\newtheorem{example}[theorem]{Example}
\newtheorem{note}[theorem]{Note}
\newtheorem{remark}[theorem]{Remark}

\newtheorem{assumption}[theorem]{Assumption}
\providecommand{\qed}{\hfill \mbox{\raggedright \rule{0.1in}{0.1in} } }

% gray proof
\let\myoldproof=\proof
\let\myoldendproof=\endproof
\renewenvironment{proof}
{\myoldproof\color{gray}}
{\myoldendproof}
\usepackage[margin=0.75in]{geometry}
\linespread{1.02}

\newcommand{\algsize}{\footnotesize}
\setlength{\floatsep}{0.0in}
\setlength{\textfloatsep}{0.0in}
\setlength{\intextsep}{0.0in}
\setlength{\belowcaptionskip}{0.05in}
\setlength{\abovecaptionskip}{0.1in}
\setlength{\abovedisplayskip}{0.05in}
\setlength{\belowdisplayskip}{0.05in}


\title{Phase transitions in random $k$-SAT}
\author{Pratik Chaudhari$^*$}
\thanks{$^*$ Laboratory of Information and Decision Systems, MIT.\newline
Email: \href{pratik.ac@gmail.com}{pratik.ac@gmail.com}}
\date{May 3, 2014}

\begin{document}
\begin{abstract}
It can be shown that there exists a sharp threshold $\alpha_c(k)$ for the ratio $m/n$ beyond which, an $m$-clause formula with $n$ variables and $k$ variables in each clause becomes unsatisfiable with high probability. The first part of this project will discuss bounds for $\alpha_c(k)$ using the second moment method.
%
Next, we discuss the phenomenon of ``dynamical phase transition'' from a statistical physics perspective, wherein the solution space in $k$-SAT transitions into the ``glassy'' phase. Using these concepts, we discuss an algorithm known as ``survey propagation'', which uses 1-step replica symmetry breaking to exploit the glassy structure of the solution space.
\end{abstract}
\maketitle

\section{Introduction}
In recent years, it has become increasingly apparent that phenomenon in statistical physics such as phase transitions and glassy phases have a strong bearing on important problems in computer science such as satisfiability, error correcting codes etc. A number of recent works such as~\cite{krzakala2012statistical} have also drawn connections between compressed sensing and statistical physics.
$$
\ones \ones
$$


We can get a simple lower bound for $r$. The number of all possible $k$-clauses is $C_k = 2^k {n \choose k}$ while the number of clauses satisfied by a given random assignment is $S_k = (2^k -1) {n \choose k}$. Thus the probability of $F \in \SAT$ is ${S_k \choose m}/{C_k \choose m} < (1-2^{-k})^m$, which means that the expected number of satisfying assignments is $2^n (1-2^{-k})^{rn} = \smallo(1)$ for $r > 2^k \log 2$ which implies $r_k^* < 2^k \log 2$.

\section{Setup}
\label{sec:setup}
A $k$-clause is a disjunction of $k$ Boolean variables, given $n$ Boolean
variables, a random $k$-CNF forumla, $F_k(n, m)$ is formed by conjunction of $m = rn$ such $k$-clauses, selected uniformly and independently with
replacement from the set of all $k$-clauses on these $n$ variables. We assume that all formulae are in CNF and in turn, this model is called $\ksat$. Also,
let $r_k = \sup \{ r: F_k(n, rm) \in \SAT \}$ and $r_k^* = \inf \{ r: F_k(n, rn) \not\in \SAT \}$. We know that $r_k < r_k^*$ and in Sec.~\ref{sec:sat_thresh}
 we will show that $r_k = r_k^*(1-\smallo(1))$. We will also show that
\beq{
r_k \geq 2^k \log2 - (k+1)\f{\log2}{2} - \bigo(1).
\label{eqn:r_k_exact}
}
Note that we always talk of $F_k(n,rn) \in \SAT$ with high probability, i.e., $\lim_{n\to \infty} \P(F_k(n,rn) \in \SAT) \to 1$ if $r < r_k$.


\section{Satisfiability threshold}
\label{sec:sat_thresh}


\subsection{2-SAT}
\label{ssec:2sat}
2-SAT has a special structure, observe that $\n{x_1} \oor x_2$ is equivalent to $x_1 \implies x_2$, i.e., given $F_2(n, rn)$, we can construct a graph with
vertices $x_1, \n{x_1}, x_2, \ldots, \n{x_n}$ and an edge from $x_1$ to $x_2$ and $\n{x_2}$ to $\n{x_1}$. It can be shown that $F_2(n,rn) \notin \SAT$ iff
there exists a path from some $x_k$ to $\n{x_k}$ and from $\n{x_k}$ to $x_k$, i.e., a bicycle. Let the length of this bicyle be $s$ if the path $(u, w_1, w_2
, \ldots, w_s, v)$ with $u,v \in \{w_1, \ldots, w_s,\n{w_1}, \ldots, \n{w_s}\}$
 exists. The probability of this happening is
$$
\P(F_2(n,rn) \notin \SAT) \leq \sum_{s=2}^n {n \choose s} s^2 (2s)^2 {m \choose s+1} \rbrac{\f{1}{4 {n \choose 2}}}^{s+1}
$$
It is easy to see that this is $\bigo(1/N)$ iff $r <1$, i.e., $r_2^* < 1$. Using Eqn.~\eqref{eqn:r_k_exact} we see that $r_2 > 1$, i.e., the phase transition threshold is simply $r = 1$.


{
\small
\bibliography{writeup}
\bibliographystyle{abbrv}
}
\end{document}
