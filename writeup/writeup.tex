\documentclass[letterpaper, 10pt, twocolumn, reqno]{amsart}

\usepackage{mathpazo}
\usepackage{amsmath, amsthm, amssymb, bm, enumerate, url, setspace}
\usepackage[usenames,dvipsnames,svgnames,table]{xcolor}
\usepackage[pdftex, xetex]{graphicx}
\usepackage[font={small}]{caption}
\usepackage{float, colortbl, tabularx, multirow, subcaption, environ, wrapfig, textcomp}
\usepackage[normalem]{ulem}

\usepackage{blindtext}

\usepackage[bookmarks=true]{hyperref}
%\usepackage[hyperpageref]{backref}
\hypersetup{
colorlinks=true, linkcolor=blue, citecolor=ForestGreen, filecolor=magenta, urlcolor=blue,
}
\usepackage{url, cite}

\newcommand{\pc}[2]{{\color{blue} #1}\marginpar{\tiny\noindent{\raggedright{\color{ForestGreen}[PC]}\color{brown}{ #2} \par}}}
\newcommand{\todo}[1]{{\color{gray}#1}\marginpar{\tiny\noindent{\raggedright{\color{ForestGreen}[MILD TODO]}}}}

%\newcommand{\myclearpage}{\clearpage}
\newcommand{\myclearpage}{}

% math macros
\newcommand{\abs}[1]{\ensuremath \left| #1 \right|}
\newcommand{\norm}[1]{\ensuremath \lVert#1\rVert}
\newcommand{\given}{\, \vert \,}
\providecommand{\cal}[1]{\ensuremath \mathcal{#1}}
\newcommand{\ag}[1]{\ensuremath \left\langle#1\right\rangle}
\providecommand\bigo{O}
\providecommand\smallo{o}

% shorcuts
\newcommand{\aeq}[1]{\begin{align} #1 \end{align}}
\newcommand{\aeqs}[1]{\begin{align*} #1 \end{align*}}
\newcommand{\beq}[1]{\begin{equation}#1\end{equation}}
\newcommand{\beqs}[1]{\begin{equation*}#1\end{equation*}}
\newcommand{\trm}[1]{\ensuremath \textrm{#1}}
\newcommand{\enum}[2]{\begin{enumerate}[#1]{#2}\end{enumerate}}
\newcommand{\clist}[1]{\begin{itemize}\setlength{\itemsep}{0pt}
\setlength{\parsep}{0pt}
{#1}\end{itemize}}
\newcommand{\bmat}[1]{\begin{bmatrix}#1\end{bmatrix}}
\newcommand{\mpage}[2]{\begin{center}
\begin{minipage}{#1}#2\end{minipage}\end{center}}
\newcommand{\la}{\leftarrow}
\newcommand{\ra}{\rightarrow}
\providecommand\f[2]{\ensuremath \frac{#1}{#2}}
\providecommand\rbrac[1]{\ensuremath \left(#1\right)}
\providecommand\sqbrac[1]{\ensuremath \left[#1\right]}
\providecommand\cbrac[1]{\ensuremath \left\{#1\right\}}

\newtheorem{theorem}{Theorem}
\newtheorem{proposition}[theorem]{Proposition}
\newtheorem{lemma}[theorem]{Lemma}
\newtheorem{corollary}[theorem]{Corollary}
\newtheorem{problem}[theorem]{Problem}

\theoremstyle{definition}
\newtheorem{definition}[theorem]{Definition}
\newtheorem{example}[theorem]{Example}
\newtheorem{note}[theorem]{Note}
\newtheorem{remark}[theorem]{Remark}

\newtheorem{assumption}[theorem]{Assumption}
\providecommand{\qed}{\hfill \mbox{\raggedright \rule{0.1in}{0.1in} } }

% gray proof
\let\myoldproof=\proof
\let\myoldendproof=\endproof
\renewenvironment{proof}
{\myoldproof\color{gray}}
{\myoldendproof}

% notation
\renewcommand{\P}{\trm{P}}
\newcommand{\E}{\trm{E}}
\providecommand{\ones}{\mathbb{1}}
\providecommand{\ind}{{\bf 1}}
\providecommand{\SAT}{\trm{SAT}}
\providecommand{\NAESAT}{\trm{NAE-SAT}}
\providecommand\sat{\models}
\providecommand\ksat{k\trm{-SAT}}
\providecommand\n[1]{\overline{#1}}
\providecommand{\oor}{\vee}
\providecommand{\aand}{\wedge}
%\providecommand{\implies}{\rightarrow}
\renewcommand{\sf}{S(F)}
\renewcommand{\sc}{S(c)}
\providecommand\s{\sigma}
\renewcommand{\t}{\tau}
\renewcommand{\a}{\alpha}
\renewcommand{\b}{\beta}
\renewcommand{\L}{\Lambda}
\renewcommand{\l}{\lambda}

\newcommand{\wsf}{w(\s,F)}
\newcommand{\lsf}{L(\s,F)}
\newcommand{\wsc}{w(\s,c)}
\newcommand{\wv}{w(v)}
\newcommand{\wu}{w(u)}
\newcommand{\wtf}{w(\t,F)}
\newcommand{\wtc}{w(\t,c)}
\newcommand{\puva}{\Phi_{u,v}(\a)}

\usepackage[margin=0.75in]{geometry}
\linespread{1.05}

\newcommand{\algsize}{\footnotesize}
\setlength{\floatsep}{0.0in}
\setlength{\textfloatsep}{0.0in}
\setlength{\intextsep}{0.0in}
\setlength{\belowcaptionskip}{0.05in}
\setlength{\abovecaptionskip}{0.1in}
\setlength{\abovedisplayskip}{0.05in}
\setlength{\belowdisplayskip}{0.05in}


\title{Phase transitions in random $k$-SAT}
\author{Pratik Chaudhari$^*$}
\thanks{$^*$ Laboratory of Information and Decision Systems, MIT.\newline
Email: \href{pratik.ac@gmail.com}{pratik.ac@gmail.com}}
\date{May 3, 2014}

\begin{document}
\begin{abstract}
This project discuses bounds for sharp transition thresholds in random satisfiability problems using the second moment method. Next, we discuss ``dynamical phase transition'', i.e., phase transition into a ``glassy'' state and motivate an algorithm known as ``survey propagation'', which uses ideas from statistical physics to exploit the glassy structure of the solution space.
\end{abstract}
\maketitle

\section{Introduction}
In recent years, it has become increasingly apparent that phenomenon in statistical physics such as phase transitions and glassy phases have a strong bearing on important problems in computer science such as satisfiability, error correcting codes etc. A number of recent works such as~\cite{krzakala2012statistical} have also drawn connections between compressed sensing and statistical physics.


\section{Setup}
\label{sec:setup}
A $k$-clause is a disjunction of $k$ Boolean variables, given $n$ Boolean
variables, a random $k$-CNF forumla, $F_k(n, m)$ is formed by conjunction of $m = rn$ such $k$-clauses, selected uniformly and independently with
replacement from the set of all $k$-clauses on these $n$ variables. We assume that all formulae are in CNF and in turn, this model is called $\ksat$. Also,
let $r_k = \sup \{ r: F_k(n, rm) \in \SAT \}$ and $r_k^* = \inf \{ r: F_k(n, rn) \not\in \SAT \}$. Note that we always talk of $F_k(n,rn) \in \SAT$ with high probability, i.e., $\lim_{n\to \infty} \P(F_k(n,rn) \in \SAT) = 1$ if $r < r_k$.

We know that $r_k < r_k^*$ and in Sec.~\ref{sec:sat_thresh}
 we will show that $r_k = r_k^*(1-\smallo(1))$. We will also show that
\beq{
r_k \geq 2^k \log2 - (k+1)\f{\log2}{2} - \bigo(1).
\label{eqn:r_k_exact}
}

\section{Satisfiability threshold}
\label{sec:sat_thresh}


\subsection{2-SAT}
\label{ssec:2sat}
2-SAT has a special structure, observe that $\n{x_1} \oor x_2$ is equivalent to $x_1 \implies x_2$, i.e., given $F_2(n, rn)$, we can construct a graph with
vertices $x_1, \n{x_1}, x_2, \ldots, \n{x_n}$ and an edge from $x_1$ to $x_2$ and $\n{x_2}$ to $\n{x_1}$. It can be shown that $F_2(n,rn) \notin \SAT$ iff
there exists a path from some $x_k$ to $\n{x_k}$ and from $\n{x_k}$ to $x_k$, i.e., a bicycle. Let the length of this bicycle be $s$ if the path $(u, w_1, w_2
, \ldots, w_s, v)$ with $u,v \in \{w_1, \ldots, w_s,\n{w_1}, \ldots, \n{w_s}\}$
 exists. The probability of this happening is
$$
\P(F_2(n,rn) \notin \SAT) \leq \sum_{s=2}^n {n \choose s} s^2 (2s)^2 {m \choose s+1} \rbrac{\f{1}{4 {n \choose 2}}}^{s+1}
$$
By direct summation, it is easy to see that this is $\bigo(1/n)$ iff $r <1$, i.e., $r_2^* < 1$. Using Eqn.~\eqref{eqn:r_k_exact} we see that $r_2 > 1$, i.e., the phase transition threshold is simply $r = 1$.

\subsection{Vanilla second moment for $\ksat$}
\label{ssec:ksat_vanilla}
To get a ball-park estimate, let us compute a simple upper bound for the phase transition. The number of all possible $k$-clauses is $C_k = 2^k {n \choose k}$ while the number of clauses satisfied by a given random assignment is $S_k = (2^k -1) {n \choose k}$. Thus the probability of $F \in \SAT$ is ${S_k \choose m}/{C_k \choose m} < (1-2^{-k})^m$, which means that the expected number of satisfying assignments is $2^n (1-2^{-k})^{rn} = \smallo(1)$ for $r > 2^k \log 2$ which implies $r_k^* < 2^k \log 2$.

The idea behind this section is to get a non-zero lower bound in the limit for $P(X > 0)$, where $X$ is the number of solutions of random $\ksat$. We can then use the inequality $P(X>0) \geq \E[X]^2/\E[X^2]$. It is however instructive to note that we cannot weigh each satisfying assignment equally, the bounds are too loose that way. Let us briefly review why.

For a random formula $F$, let $\sf$ be the set of satisfying assignments and let $X = \abs{\sf}$. If $F = c_1 \aand c_2 \ldots c_m$, we have
\aeqs{
\E[X^2] &= \E\sqbrac{\rbrac{\sum_\s \ind_{\s \in \sf}}^2} &&= \E\sqbrac{\sum_{\s,\t} \ind_{\s,\t \in \sf}} \\
&= \sum_{\s,\t} \E \sqbrac{\prod_c \ind_{\s,\t \in \sc}} &&= \sum_{\s,\t} \prod_c \E[\ind_{\s,\t \in \sc}].
}
We can now see that
$$
\P(\s,\t \in \sc) = 1 -2^{1-k} - 2^{-k}\a^k = f_s(\a)
$$
where $\s$ and $\t$ assign the same value to $z = \a n$ variables. This follows because if $\s \not\sat c$, the only way for $\t \not\in c$ is for all $k$ variables in $c$ to lie in the overlap. $f_S$ thus quantifies the correlation between $\s,\t$ being satisfying. We now have
$$
\E[X^2] = 2^n \sum_{z=0}^n {n \choose z} f_S(\a)^m,
$$
whereupon, using ${n \choose z} = \rbrac{\a^\a (1-\a)^{1-\a}}^{-n} poly(n)$, we have
\aeqs{
\E[X^2] &\leq 2^n \rbrac{\max_{0 \leq \a \leq 1} \sqbrac{ \f{f_S(\a)^r}{\a^\a (1-\a)^{1-\a}} } }^n poly(n) \\
&:= \rbrac{\max_{0 \leq \a \leq 1} \L_S(\a)}^n poly(n).
}

At the same time, note that
$$
\E[X]^2 = \rbrac{2^n \rbrac{1 - 2^{-k}}^m }^2 = \L_S(1/2)^n,
$$
and hence even if we have some $\a \in [0,1]$ with $\L_S(\a) > \L_S(1/2)$ the
the second moment is exponentially greater than the square of the first moment and hence we have an exponentially smaller lower bound on $\P(X > 0)$. In
other words, we need to reduce the contribution to $E[X^2]$ from $\s,\t$ with overlap of $\a > 1/2$. Work on Not All Equal-SAT, i.e., every clause has at
least one satisfied and one unsatisfied literal gives some clues. For $\NAESAT$, we have
$$
\P(\s,\t \in \NAESAT) = 1 - 2^{2-k} + 2^{1-k} \rbrac{\a^k + (1-\a)^k}
$$
which is symmetric about $\a = 1/2$ and as a result, $\L_N(\a)$ has a maxima at $1/2$ for $r \leq 2^{k-1} \log2 -1$. It was shown in~\cite{achlioptas2002asymptotic} that for $r \geq 2^{k-1} \log2$, $\NAESAT$ is not satisfiable, i.e., the second moment method actually gives bounds within additive constant.

\subsection{Weighted second moment for $\ksat$}
\label{ssec:ksat_weighted}
We next use this intuition to construct a weighted second moment method. Let $\wsf$ be some function and $\wsf = 0$ if $\s \notin \sf$. Let $\wsf$ factor exactly over the clauses to give
$$X = \sum_\s \wsf = \sum_\s \prod_c \wsc$$
which still works for us because if we can prove $\E[X^2] = \bigo\rbrac{\E[X]^2}$, we will still have $\P(X > 0) = \P(\abs{\sf} > 0)$. Again,
\aeqs{
\E[X]^2 &= 2^n \rbrac{\E[\wsc]}^m \\
\E[X^2] %&= \sum_{\s,\t} \prod_c \E[\wsc \wtc]
&= \sum_{\s,\t} \rbrac{\E[\wsc \wtc]}^m
}
where the clause $c = \ell_1 \aand \ldots \aand \ell_k$ is random. We want $\wsc$ to be independent of variable labels, so let it be $\wsc = \wv$ where $v_i =1$ if $\s \sat \ell_i$ and $-1$ otherwise.

$v \in A = \{-1,1\}^k$ and since the literals are drawn uniformly and independently, we have
$$
\E[\wsc] = \sum_{v \in A} \wv 2^{-k}.
$$
Similarly for $\s, \t$ with an overlap of $z =\a n$,
\aeqs{
\E[\wsc \wtc] &= \sum_{u,v \in A} \wu \wv 2^{-k} \prod_{i=1}^k \a^{\ind_{u_i = v_i}} (1-\a)^{\ind_{u_i \neq v_i}} \\
&:= \sum_{u,v \in A} \wu \wv \puva := f_w(\a).
}
Again, we have
\aeqs{
\E[X^2] &= 2^n \sum_{z=0}^n {n \choose z} f_w(\a)^m \\
&\leq 2^n \rbrac{\max_{0 \leq \a \leq 1} \sqbrac{ \f{f_w(\a)^r}{\a^\a (1-\a)^{1-\a}} } }^n poly(n) \\
&:= \rbrac{\max_{0 \leq \a \leq 1} \L_w(\a)}^n poly(n)
}
and $\E[X]^2 = \L_w(1/2)^n$. There are two aspects of the above expression which will help us get $\E[X^2]/\E[X]^2 = \bigo(1)$. $\L_w(\a)$ has a global maximum at $1/2$ and the $poly(n)$ factor above is instead $\bigo(1)$. The former requires that $f_w'(1/2) =0$. We thus have
\aeqs{
f_w'(\a) &= \sum_{u,v \in A} \wu \wv \puva \sqbrac{\log \puva}' \\
&= \sum_{u,v \in A} \wu \wv \puva \sum_{i=1}^k\ \rbrac{\f{\ind_{u_i = v_i}}{\a} - \f{\ind_{u_i \neq v_i}}{1-\a}} \\
2^{2k-1} f_w'(\a) &= \sum_{u,v \in A} \wu \wv u v \\
&= \rbrac{\sum_u \wu u}\rbrac{\sum_v \wv v}.
}
Therefore for any $w$, we need
\beq{
f_w'(1/2) = 0 \implies \sum_{v \in A} \wv v = 0.
\label{eqn:wv_sum}
}

It is now evident that the real reason why the calculations in Sec.~\ref{ssec:ksat_vanilla} failed was because $w_S(\cdot)$ assigns 0 to $(-1,\ldots,-1)$
and $1/(2^k-1)$ to all other vectors. On the other hand, for $\NAESAT$, $w_N(\cdot)$ assigns 0 to both $(-1,\ldots,-1)$ and $(1,\ldots,1)$ and hence satisfies Eqn.~
\eqref{eqn:wv_sum}. Hence, we shall construct such a $w(\cdot)$. In addition to this, to sharpen the results, we need to make sure that $w$ is as close $w_S
$ as possible. By a simple summation, Eqn.~\eqref{eqn:wv_sum} can also be written as
$$
\sum_{v \neq (-1,\ldots,-1)} \wv (2\abs{v} - k) = 0.
$$
where $\abs{v}$ is the number of 1s in $v$. We maximize entropy ($w_S$ is uniform and has maximum entropy) subject to this constraint to get $\wv \propto \l^{\abs{v}}$ with $\l$ satisfying $(1+\l)^{k-1} = 1/(1-\l)$.

We therefore work with a weighing function
$$
\wsf \propto \prod_c \l^{L(\s,F)} \ind_{\s \in \sc}
$$
where $\lsf$ is the number of literals in $F$ satisfied by $\s$. This results in the following theorem.

\begin{theorem}
There exists a sequence $\b_k \to 0$ s.t. for all $k \geq 3$,
$$
r_k \geq 2^k \log 2 - 2(k+1) \log 2 - 1 - \b_k.
$$
\end{theorem}
\begin{remark}
Note that the second term is off by a factor of 4 as compared to Eqn.~\eqref{eqn:r_k_exact}. This is because of our constraint that $\wsf$ factor exactly over all clauses. It is possible to sharpen these bounds as shown in Sec.~\ref{ssec:sharper_bounds}. \todo{Also, in the following proof, we will neglect improper clauses, i.e., repeated or contradictory literals since there probability is at most $k^2/n = \smallo(n)$ of them. Similarly, we also let the model select clauses with replacement since w.h.p there are at most $\smallo(n)$ clauses that contain the same $k$ variables.}
\end{remark}
%
\begin{proof}

\end{proof}

\subsection{Sharper bounds using measure transform}
\label{ssec:sharper_bounds}

\begin{remark}
\todo{talk about the ``mist'' in the solution space}
\end{remark}

\section{Dynamical phase transition}
\label{sec:dynamical_transition}

We describe a few results about the geometry of the solution space of random $\ksat$. The motivation for these results comes from the following --- a very
simple algorithm can find satisfying assignments for $r = \bigo(2^k/k)$ with
high probability, it simply assigns a random value to a randomly unassigned variable and simplifies the clauses. In fact an algorithm based on belief
propagation can solve instances upto $r = \bigth(2^k \log k/k)$ using an idea known as ``decimation''~\cite{montanari2007solving}. In Sec.~\ref{sec:sat_thresh}, we saw that $\ksat$ is in fact satisfiable for far more after that; it has solutions (w.h.p) until $r = 2^k \log 2 - O(k)$. This apparent discrepancy can
be resolved with the following theorem towards which we work in this section.
\begin{theorem}
There exists a sequence $\e_k \to 0$ such that $\sf$ shatters for all
$$
(1 + \e_k) \f{2^k}{k} \log k \leq r \leq (1-\e_k) 2^k \log 2.
$$
\label{thm:shatter}
\end{theorem}
%
It turns out that the solution space, i.e., $\sf$ looks like a giant ball
until $r = \bigo(2^k/k)$ and shatters into exponentially many clusters with
exponentially small number of solutions in each as $r$ increases. Towards that end, let us define a few notions about the solution space.
\clist{
    \item We will use the Hamming distance between two satisfying instances $\s,\t$ and say that they are adjacent if $\abs{\s-\t} = 1$. \pc{A region is then the union of connected components of $\sf$.}{?}
    \item The set of solutions $\sf$ ``shatters'' if there exist $\b, \g, \zeta > 0$ s.t. w.h.p,
    \clist{
    \item $\sf$ can be partitioned into at least $e^{\b n}$ regions with each region containing at most $e^{\g n}$ fraction of solutions;
    \item any two regions are at least $\zeta n$ apart;
    \item  every path between two solutions $\s, \t$ in distinct regions has a ``height'' at least $\th n$, i.e., maximum number of clauses
    violated by any sequence $\s, \s_1, \ldots, \s_n, \t$ with at most unit distance between consecutive elements.
    }
}


\section{1-step replica symmetry breaking}
\label{sec:1rsb}

\section{Summary}
\label{sec:summary}



{
\small
\bibliography{writeup}
\bibliographystyle{abbrv}
}
\end{document}
