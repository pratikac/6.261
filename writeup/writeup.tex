\documentclass[letterpaper, 10pt, twocolumn, reqno]{amsart}

\usepackage{mathpazo}
\usepackage{amsmath, amsfonts, amssymb, bm, enumerate, url, setspace}
\usepackage[boxruled, vlined, linesnumbered]{algorithm2e}
\usepackage[usenames,dvipsnames,svgnames,table]{xcolor}
\usepackage[pdftex, xetex]{graphicx}
\usepackage[font={small}]{caption}
\usepackage{float, colortbl, tabularx, multirow, subcaption, environ, wrapfig, textcomp}
\usepackage[normalem]{ulem}

\usepackage{pgf, tikz, framed}
%\usetikzlibrary{arrows,positioning,automata,shadows,fit,shapes}

\usepackage[english]{babel}
\usepackage{blindtext}

\usepackage[bookmarks=true]{hyperref}
\usepackage[hyperpageref]{backref}
\hypersetup{
colorlinks=true, linkcolor=blue, citecolor=darkgreen, filecolor=magenta, urlcolor=blue,
}
\usepackage{url, cite}

\newcommand{\pc}[2]{{\color{blue} #1}\marginpar{\tiny\noindent{\raggedright{\color{darkgreen}[PC]}\color{brown}{ #2} \par}}}
\newcommand{\mtodo}[1]{{\color{gray}#1}\marginpar{\tiny\noindent{\raggedright{\color{darkgreen}[MILD TODO]}}}}

%\newcommand{\myclearpage}{\clearpage}
\newcommand{\myclearpage}{}

% math macros
\newcommand{\abs}[1]{\ensuremath \left| #1 \right|}
\newcommand{\norm}[1]{\ensuremath \lVert#1\rVert}
\newcommand{\given}{\, \vert \,}
\renewcommand{\cal}[1]{\ensuremath \mathcal{#1}}
\newcommand{\ag}[1]{\ensuremath \left\langle#1\right\rangle}

% shorcuts
\newcommand{\aeq}[1]{\begin{align} #1 \end{align}}
\newcommand{\aeqs}[1]{\begin{align*} #1 \end{align*}}
\newcommand{\beq}[1]{\begin{equation}#1\end{equation}}
\newcommand{\beqs}[1]{\begin{equation*}#1\end{equation*}}
\newcommand{\trm}[1]{\ensuremath \textrm{#1}}
\newcommand{\enum}[2]{\begin{enumerate}[#1]{#2}\end{enumerate}}
\newcommand{\ilist}[1]{\begin{itemize}{#1}\end{itemize}}
\newcommand{\clist}[1]{\begin{itemize}\setlength{\itemsep}{0pt}
\setlength{\parsep}{0pt}
{#1}\end{itemize}}
\newcommand{\bmat}[1]{\begin{bmatrix}#1\end{bmatrix}}
\newcommand{\mpage}[2]{\begin{center}
\begin{minipage}{#1}#2\end{minipage}\end{center}}
\newcommand{\la}{\leftarrow}
\newcommand{\ra}{\rightarrow}

% theorems
\usepackage{theorem}
\newtheorem{theorem}{Theorem}

\newtheorem{proposition}[theorem]{Proposition}
\newtheorem{lemma}[theorem]{Lemma}
\newtheorem{corollary}[theorem]{Corollary}
\newtheorem{problem}[theorem]{Problem}

\theoremstyle{definition}
\newtheorem{definition}[theorem]{Definition}
\newtheorem{example}[theorem]{Example}
\newtheorem{note}[theorem]{Note}
\newtheorem{remark}[theorem]{Remark}

\newtheorem{assumption}[theorem]{Assumption}
\providecommand{\qed}{\hfill \mbox{\raggedright \rule{0.1in}{0.1in} } }

% gray proof
\let\myoldproof=\proof
\let\myoldendproof=\endproof
\renewenvironment{proof}
{\myoldproof\color{gray}}
{\myoldendproof}
\usepackage[margin=0.75in]{geometry}
\linespread{1.05}

\newcommand{\algsize}{\footnotesize}
\setlength{\floatsep}{0.0in}
\setlength{\textfloatsep}{0.0in}
\setlength{\intextsep}{0.0in}
\setlength{\belowcaptionskip}{0.05in}
\setlength{\abovecaptionskip}{0.1in}
\setlength{\abovedisplayskip}{0.05in}
\setlength{\belowdisplayskip}{0.05in}

\graphicspath{{../fig/}}

\title{Phase transitions in random $k$-SAT}
\author{Pratik Chaudhari$^*$}
\thanks{$^*$ Laboratory of Information and Decision Systems, MIT.\newline
Email: \href{pratik.ac@gmail.com}{pratik.ac@gmail.com}}
\date{May 3, 2014}

\begin{document}
\maketitle
{\small
\textbf{\emph{Abstract:}}
In recent years, it has become increasingly apparent that phenomenon in statistical physics such as phase transitions and glassy phases have a strong bearing on important problems in computer science such as satisfiability, error correcting codes etc. A number of recent works such as~\cite{krzakala2012statistical} have also drawn connections between compressed sensing and statistical physics.

This project discuses bounds for sharp transition thresholds in random satisfiability problems using the second moment method. Next, we discuss ``dynamical phase transition'', i.e., phase transition into a ``glassy'' state and motivate an algorithm known as ``survey propagation'', which uses ideas from statistical physics to exploit the glassy structure of the solution space.
}

% \section{Introduction}
% In recent years, it has become increasingly apparent that phenomenon in statistical physics such as phase transitions and glassy phases have a strong bearing on important problems in computer science such as satisfiability, error correcting codes etc. A number of recent works such as~\cite{krzakala2012statistical} have also drawn connections between compressed sensing and statistical physics.


\section{Setup}
\label{sec:setup}
A $k$-clause is a disjunction of $k$ Boolean variables, given $n$ Boolean
variables, a random $k$-CNF forumla, $F_k(n, m)$ is formed by conjunction of $m = rn$ such $k$-clauses, selected uniformly and independently with
replacement from the set of all $k$-clauses on these $n$ variables. We assume that all formulae are in CNF and in turn, this model is called $\ksat$. Also,
let $r_k = \sup \{ r: F_k(n, rm) \in \SAT \}$ and $r_k^* = \inf \{ r: F_k(n, rn) \not\in \SAT \}$. Note that we always talk of $F_k(n,rn) \in \SAT$ with high probability, i.e., $\lim_{n\to \infty} \P(F_k(n,rn) \in \SAT) = 1$ if $r < r_k$.

We know that $r_k < r_k^*$ and in Sec.~\ref{sec:sat_thresh}
we will show that $r_k = r_k^*(1-\smallo(1))$. We will also show that
\beq{
r_k \geq 2^k \log2 - (k+1)\f{\log2}{2} - \bigo(1).
\label{eqn:r_k_exact}
}

\section{Satisfiability thresholds}
\label{sec:sat_thresh}


\subsection{2-SAT}
\label{ssec:2sat}
2-SAT has a special structure, observe that $\n{x_1} \oor x_2$ is equivalent to $x_1 \implies x_2$, i.e., given $F_2(n, rn)$, we can construct a graph with
vertices $x_1, \n{x_1}, x_2, \ldots, \n{x_n}$ and an edge from $x_1$ to $x_2$ and $\n{x_2}$ to $\n{x_1}$. It can be shown that $F_2(n,rn) \notin \SAT$ iff
there exists a path from some $x_k$ to $\n{x_k}$ and from $\n{x_k}$ to $x_k$, i.e., a bicycle. Let the length of this bicycle be $s$ if the path $(u, w_1, w_2
, \ldots, w_s, v)$ with $u,v \in \{w_1, \ldots, w_s,\n{w_1}, \ldots, \n{w_s}\}$
exists. The probability of this happening is
$$
\P(F_2(n,rn) \notin \SAT) \leq \sum_{s=2}^n {n \choose s} s^2 (2s)^2 {m \choose s+1} \rbrac{\f{1}{4 {n \choose 2}}}^{s+1}
$$
By direct summation, it is easy to see that this is $\bigo(1/n)$ iff $r <1$, i.e., $r_2^* < 1$. Using Eqn.~\eqref{eqn:r_k_exact} we see that $r_2 > 1$, i.e., the phase transition threshold is simply $r = 1$.

\subsection{Vanilla second moment for $\ksat$}
\label{ssec:ksat_vanilla}
To get a ball-park estimate, let us compute a simple upper bound for the phase transition. The number of all possible $k$-clauses is $C_k = 2^k {n \choose k}$ while the number of clauses satisfied by a given random assignment is $S_k = (2^k -1) {n \choose k}$. Thus the probability of $F \in \SAT$ is ${S_k \choose m}/{C_k \choose m} < (1-2^{-k})^m$, which means that the expected number of satisfying assignments is $2^n (1-2^{-k})^{rn} = \smallo(1)$ for $r > 2^k \log 2$ which implies $r_k^* < 2^k \log 2$.

The idea behind this section is to get a non-zero lower bound in the limit for $P(X > 0)$, where $X$ is the number of solutions of random $\ksat$. We can then use the inequality $P(X>0) \geq \E[X]^2/\E[X^2]$. It is however instructive to note that we cannot weigh each satisfying assignment equally, the bounds are too loose that way. Let us briefly review why.

For a random formula $F$, let $\sf$ be the set of satisfying assignments and let $X = \abs{\sf}$. If $F = c_1 \aand c_2 \ldots c_m$, we have
\aeq{
\E[X^2] &= \E\sqbrac{\rbrac{\sum_\s \ind_{\s \in \sf}}^2} &&= \E\sqbrac{\sum_{\s,\t} \ind_{\s,\t \in \sf}} \notag \\
&= \sum_{\s,\t} \E \sqbrac{\prod_c \ind_{\s,\t \in \sc}} &&= \sum_{\s,\t} \prod_c \E[\ind_{\s,\t \in \sc}].
\label{eqn:calc_x2}
}
We can now see that
$$
\P(\s,\t \in \sc) = 1 -2^{1-k} - 2^{-k}\a^k = f_s(\a)
$$
where $\s$ and $\t$ assign the same value to $z = \a n$ variables. This follows because if $\s \not\sat c$, the only way for $\t \not\in c$ is for all $k$ variables in $c$ to lie in the overlap. $f_S$ thus quantifies the correlation between $\s,\t$ being satisfying. We now have
$$
\E[X^2] = 2^n \sum_{z=0}^n {n \choose z} f_S(\a)^m,
$$
whereupon, using ${n \choose z} = \rbrac{\a^\a (1-\a)^{1-\a}}^{-n} poly(n)$, we have
\aeq{
\E[X^2] &\leq 2^n \rbrac{\max_{0 \leq \a \leq 1} \sqbrac{ \f{f_S(\a)^r}{\a^\a (1-\a)^{1-\a}} } }^n poly(n) \notag \\
&:= \rbrac{\max_{0 \leq \a \leq 1} \L_S(\a)}^n poly(n).
\label{eqn:Ls}
}

At the same time, note that
$$
\E[X]^2 = \rbrac{2^n \rbrac{1 - 2^{-k}}^m }^2 = \L_S(1/2)^n,
$$
and hence even if we have some $\a \in [0,1]$ with $\L_S(\a) > \L_S(1/2)$ the
the second moment is exponentially greater than the square of the first moment and hence we have an exponentially smaller lower bound on $\P(X > 0)$. In
other words, we need to reduce the contribution to $E[X^2]$ from $\s,\t$ with overlap of $\a > 1/2$. Work on Not All Equal-SAT, i.e., every clause has at
least one satisfied and one unsatisfied literal gives some clues. For $\NAESAT$, we have
$$
\P(\s,\t \in \NAESAT) = 1 - 2^{2-k} + 2^{1-k} \rbrac{\a^k + (1-\a)^k}
$$
which is symmetric about $\a = 1/2$ and as a result, $\L_N(\a)$ has a maxima at $1/2$ for $r \leq 2^{k-1} \log2 -1$. It was shown in~\cite{achlioptas2002asymptotic} that for $r \geq 2^{k-1} \log2$, $\NAESAT$ is not satisfiable, i.e., the second moment method actually gives bounds within additive constant.

\subsection{Weighted second moment for $\ksat$}
\label{ssec:ksat_weighted}
We next use this intuition to construct a weighted second moment method. Let $\wsf$ be some function and $\wsf = 0$ if $\s \notin \sf$. Let $\wsf$ factor exactly over the clauses to give
$$X = \sum_\s \wsf = \sum_\s \prod_c \wsc$$
which still works for us because if we can prove $\E[X^2] = \bigo\rbrac{\E[X]^2}$, we will still have $\P(X > 0) = \P(\abs{\sf} > 0)$. Again,
\aeqs{
\E[X]^2 &= 2^n \rbrac{\E[\wsc]}^m \\
\E[X^2] %&= \sum_{\s,\t} \prod_c \E[\wsc \wtc]
&= \sum_{\s,\t} \rbrac{\E[\wsc \wtc]}^m
}
where the clause $c = \ell_1 \oor \ldots \oor \ell_k$ is random. We want $\wsc$ to be independent of variable labels, so let it be $\wsc = \wv$ where $v_i =1$ if $\s \sat \ell_i$ and $-1$ otherwise.

Let $v \in A = \{-1,1\}^k$ and since the literals are drawn uniformly and independently, we have
$$
\E[\wsc] = \sum_{v \in A} \wv 2^{-k}.
$$
Similarly for $\s, \t$ with an overlap of $z =\a n$,
\aeqs{
\E[\wsc \wtc] &= \sum_{u,v \in A} \wu \wv 2^{-k} \prod_{i=1}^k \a^{\ind_{u_i = v_i}} (1-\a)^{\ind_{u_i \neq v_i}} \\
&:= \sum_{u,v \in A} \wu \wv \puva := f_w(\a).
}
Again, we have
\aeq{
\E[X^2] &= 2^n \sum_{z=0}^n {n \choose z} f_w(\a)^m \notag \\
&\leq 2^n \rbrac{\max_{0 \leq \a \leq 1} \sqbrac{ \f{f_w(\a)^r}{\a^\a (1-\a)^{1-\a}} } }^n poly(n) \notag \\
&:= \rbrac{\max_{0 \leq \a \leq 1} \L_w(\a)}^n poly(n)
\label{eqn:Lw}
}
and $\E[X]^2 = \L_w(1/2)^n$. There are two aspects of the above ex:pression which will help us get $\E[X^2]/\E[X]^2 = \bigo(1)$. $\L_w(\a)$ has a global maximum at $1/2$ and the $poly(n)$ factor above is instead $\bigo(1)$. The former requires that $f_w'(1/2) =0$. We thus have
\aeqs{
f_w'(\a) &= \sum_{u,v \in A} \wu \wv \puva \sqbrac{\log \puva}' \\
&= \sum_{u,v \in A} \wu \wv \puva \sum_{i=1}^k\ \rbrac{\f{\ind_{u_i = v_i}}{\a} - \f{\ind_{u_i \neq v_i}}{1-\a}} \\
2^{2k-1} f_w'(\a) &= \sum_{u,v \in A} \wu \wv u v \\
&= \rbrac{\sum_u \wu u}\rbrac{\sum_v \wv v}.
}
Therefore for any $w$, we need
\beq{
f_w'(1/2) = 0 \implies \sum_{v \in A} \wv v = 0.
\label{eqn:wv_sum}
}

It is now evident that the real reason why the calculations in Sec.~\ref{ssec:ksat_vanilla} failed was because $w_S(\cdot)$ assigns 0 to $(-1,\ldots,-1)$
and $1/(2^k-1)$ to all other vectors. On the other hand, for $\NAESAT$, $w_N(\cdot)$ assigns 0 to both $(-1,\ldots,-1)$ and $(1,\ldots,1)$ and hence satisfies Eqn.~
\eqref{eqn:wv_sum}. Hence, we shall construct such a $w(\cdot)$. In addition to this, to sharpen the results, we need to make sure that $w$ is as close $w_S
$ as possible. By a simple summation, Eqn.~\eqref{eqn:wv_sum} can also be written as
$$
\sum_{v \neq (-1,\ldots,-1)} \wv (2\abs{v} - k) = 0.
$$
where $\abs{v}$ is the number of 1s in $v$. We maximize entropy ($w_S$ is uniform and has maximum entropy) subject to this constraint to get $\wv \propto \l^{\abs{v}}$ with $\l$ satisfying $(1+\l)^{k-1} = 1/(1-\l)$.

We therefore work with a weighing function
$$
\wsf \propto \prod_c \l^{L(\s,F)} \ind_{\s \in \sc}
$$
where $\lsf$ is the number of literals in $F$ satisfied by $\s$. This results in the following theorem.

\begin{theorem}
There exists a sequence $\b_k \to 0$ s.t. $\forall\ k \geq 22$,
$$
r_k \geq 2^k \log 2 - 2(k+1) \log 2 - 1 - \b_k.
$$
\label{thm:bound_weighted}
\end{theorem}
\begin{remark}
Note that the second term is off by a factor of 4 as compared to Eqn.~\eqref{eqn:r_k_exact}. This is because of our constraint that $\wsf$ factor exactly over all clauses. It is possible to sharpen these bounds as shown in Sec.~\ref{ssec:sharper_bounds}. Also, in the following proof, we will neglect improper clauses, i.e., repeated or contradictory literals since there probability is at most $k^2/n = \smallo(n)$ of them. Similarly, we also let the model select clauses with replacement since w.h.p there are at most $\smallo(n)$ clauses that contain the same $k$ variables.
\end{remark}
%
\begin{proof}[Sketch] Let $\hsf$ be the number of satisfied literal occurances in $F$ with $\s$ less the number of unsatisfied literals. Note that $\hsf = 2\lsf - k m$. For $0 < \g \leq 1$, let $X = \sum_\s \g^{\hsf} \ind_{\s \in \sf}$. Fix $\s$ and since liteals in $c = l_1 \oor \ldots \oor l_k$ are random, we have
\aeqs{
\E \sqbrac{\g^\hsc \ind_{\s \in \sf}} &= \E\sqbrac{\g^\hsc} - \E\sqbrac{\g^{-k}\ind_{\s \notin \sc}}\\
&:=\psi(\g).\\
\implies \E[X] &= (2\psi(\g)^r)^n.
}
Similarly, if $\s, \t$ overlap in $\a n$ literals, we can get
\aeq{
\E\sqbrac{\g^{\hsl + \htl}} &= \a \rbrac{\g^2 + \g^{-2}}/2 + 1 - \a \notag\\
\E\sqbrac{\g^{\hsl + \htl} \ind_{\s \notin \sc}} &= 2^{-k} (\a \g^{-2} + (1-\a)) \notag\\
\E\sqbrac{\g^{\hsl + \htl} \ind_{\s,\t \notin \sc}} &= 2^{-k} (\a \g^{-2}) \notag\\
\implies
\E\sqbrac{\g^{\hsc + \htc} \ind_{\s,\t \in \sc}} &= \f{f(\a)}{2^k(1-\e)^k}.
\label{eqn:g_st}
}
for some explicit $f(\a)$ and $\e = 1-\g^2$. Using the same calculations as in Eqn.~\eqref{eqn:calc_x2}, we get
\beq{
\E[X^2] = 2^n \sum_{z=0}^n {n \choose z} \rbrac{\f{f(\a)}{2^k(1-\e)^k}}^n.
\label{eqn:x2_weighted}
}
The right hand side can be bounded by noting that only $\bigth(n^{1/2})$ of binomial coefficients dominate, i.e., if $g = \f{f(\a)}{\a^\a + (1-\a)^{1-\a}}$,
and there exists $\a_{\max}$ s.t. $g(\a_{\max}) > g(\a)$ for all $\a \in (0,1)$ and $g''(\a_{\max}) < 0$; these conditions are true in our case for $k > 22$ and
$$
r < 2^k \log 2 -2 \log2(k+1) - 1 -3/k,
$$
then there exists $C > 0$ s.t.
$$
\E[X^2] < C\ \rbrac{\f{2 g(1/2)}{(2-2\e)^{kr}} }^n.
$$
where $\e = 1-\g^2$. Now observe that
\beq{
\E[X]^2 = \rbrac{\f{2 g(1/2)}{(2-2\e)^{kr}}}^{n}
\label{eqn:weighted_ex_explicit}
}
and the proof is complete by Chebyshev's inequality--
$$
\P(X > 0) \geq \f{\E[X]^2}{\E[X^2]} \geq \f{1}{C}.
$$
\end{proof}

\subsection{Sharper bounds using measure transform}
\label{ssec:sharper_bounds}

This section hunts down the missing factor of 4 in Thm.~\ref{thm:bound_weighted}. We note that the dominant contributions to $\E[X^2]$ come from pairs where fewer than half their literals satisfied. Hence we construct
$$\spf = \cbrac{\s \in \sf: \hsf > 0}$$
(cf. proof of Thm.~\ref{thm:bound_weighted}). In Lem.~\ref{lem:dyn_moments_close}, we will show that $\E[X_+]/\E[X] \to 1/2$ a.s. where $X_+ = \sum_{\s \in \spf}\ \g^\hsf$. If this is true, consider Eqn.~\ref{eqn:g_st} with some $\e = 1 -\th^2$ with $\th^2 \geq \g^2$, we have
\aeqs{
\E\sqbrac{\g^{\hsf + \htf} \ind_{\s,\t \in \spf}} &\leq \E\sqbrac{\th^{\hsf + \htf} \ind_{\s,\t \in \spf}} \\
&\leq \E\sqbrac{\g^{\hsf + \htf} \ind_{\s,\t \in \sf}}\\
&:= f_2(\a, \e)^m.
}
where $f_2(\a,\e)$ is the same as $f(\a) (2-2\e)^{-k}$ from Eqn.~\eqref{eqn:g_st}. Since this holds for any $\e \leq 1 -\g^2$, we can write for $\e_0 = 1 - \g^2$
\aeqs{
\E\sqbrac{\g^{\hsf + \htf} \ind_{\s,\t \in \spf}} &\leq \sqbrac{\inf_{\e \leq \e_0} f_2(\a, \e)}^m \\
\implies \E[X_+^2] &\leq 2^n \sum_{z=0}^n {n \choose z} \sqbrac{\inf_{\e \leq \e_0} f_2(\a, \e)}^m
}
Again truncate the binomial coefficients and note that from Lem.~\ref{lem:dyn_moments_close} and Eqn.~\eqref{eqn:weighted_ex_explicit}, we have
$$
5 \E[X_+^2] > \E[X]^2 = g(1/2, \e_0)^n.
$$
If we can now find a piecewise constant function $\xi$ such that $g(1/2,\e_0) > f_2(\a,\xi(\e))/(\a^\a + (1-\a)^{1-\a})$ we can again Chebyshev's inequality to the piecewise constant pieces to get $\P(X_+ > 0)$ with a new bound
$$r_k > 2^k \log 2 -\f{\log 2}{2} (k+1) -1 -50k^3 2^{-k}.$$
Please see Lem.~9 in~\cite{achlioptas2004threshold} for a proof of this idea.

\begin{lemma}
For some $\g$, as $n\to \infty$, we have $\f{\E[X_+]}{\E[X]} \to 1/2$.
\label{lem:dyn_moments_close}
\end{lemma}
\begin{proof}
This is proved by a classic measure tilting argument. Let $\bP$ be the probability assigned to a sequence of random literals that form $F$, i.e., $l_1, \ldots, l_{km}$. Tilt this measure to ensure $\E_\g[\hsc\ \ind_{\s \in \sc}] = 0$ by using
$$
\bPg[\hsl = 1] = \f{\g}{\g + \g^{-1}} = \f{2 \g}{\g + \g^{-1}} \bP[\hsl = 1].
$$
Because literals are independent, for a clause $c$, we have
$$
\bPg(c) = \f{2^k \g^{\hsc} \bP(c)}{(\g +\g^{-1})^k}
$$
\end{proof}


\section{Dynamical phase transition}
\label{sec:dynamical_transition}

We describe a few results about the geometry of the solution space of random $\ksat$. The motivation for these results comes from the following --- a very
simple algorithm can find satisfying assignments for $r = \bigo(2^k/k)$ with
high probability, it simply assigns a random value to a randomly unassigned variable and simplifies the clauses. In fact an algorithm based on belief
propagation can solve instances upto $r = \bigth(2^k \log k/k)$ using an idea known as ``decimation''~\cite{montanari2007solving}. In Sec.~\ref{sec:sat_thresh}, we saw that $\ksat$ is in fact satisfiable for far more after that; it has solutions (w.h.p) until $r = 2^k \log 2 - O(k)$. This apparent discrepancy can
be resolved with the following theorem towards which we work in this section.

It turns out that the solution space, i.e., $\sf$ looks like a giant ball
until $r = \bigo(2^k/k)$ and shatters into exponentially many clusters with
exponentially small number of solutions in each as $r$ increases. We will use the Hamming distance between two satisfying instances $\s,\t$ and say that they are adjacent if $\abs{\s-\t} = 1$. A region is then the union of connected components of $\sf$.

\begin{theorem}
For $0 < \d < 1/3$ and $r = (1-\d) 2^k \log 2$, for all $k > k_0(\d)$, there exist $\a < \b < 1/2$ such that $\sf$ consists of $2^{\e_k n}$, with $\e_k = \d/2 - 3/k^2$, non-empty cluster regions with diameter at most $\a n$ and distance between every pair of regions at least $\b n$.
\label{thm:shatter}
\end{theorem}
%

\subsection{Exponentially-many clusters}
\label{ssec:exp_many_clusters}
Note that in Eqn.~\eqref{eqn:Ls}, $\L_S(\a) = \E[\abs{\sf}]$ with overlap ratio of $\a$. For convenience, we introduce $\bL = \L_S(1-\a)$, i.e., expected number of pairs of solutions at a Hamming distance of $\a n$. The program is as follows: If there exists a $z$ s.t. there are no pairs of assignments at distance $z$, it is an upper bound for the diameter of a cluster. ALso, if we can find $\bL <1$ in an interval $(\a, \b)$, it is immediate that $\sf$ can be paritioned into regions (i.e., sets of clusters) with distance at least $\b n$ and diameter at most $\a n$.

We now show that there are exponentially-many such regions. In Sec.~\ref{ssec:ksat_weighted}, we showed that $\E[X^2] < C \max_{\a \in [0,1]} \L_w(\a)^n$ and since from the second moment method, we had $\E[X^2] < C\ \E[X]^2$; use the Payley-Zigmund inequality for $t \leq \E[X]$ to get
$$
\P(X > t) \geq \f{(\E[X]-t)^2}{\E[X^2]}.
$$
Now take $t = \E[X]/poly(n)$ and see that $X$ is within a polynomial factor of its expectation, i.e., $\L_w(1/2)^{n/2}$ with constant probability. It turns out that the event ``F has more than q solutions'' has a sharp threshold~\cite{achlioptas2008algorithmic} which implies that for $r < 2^k \log 2 - k$, $\abs{\sf} > \L_w(1/2)^{n/2}/poly(n)$ w.h.p.

We now divide this by the upper bound of the cluster diameter, let $\D = \inf \cbrac{\a: \bL < 1}$ and $g_k = \max_{\a \in [0,\D]}\ \bL(\a,k)$. The expected number of pairs of solutions with distance \emph{at most} $\D n$ is then $B < poly(n)\ g_k^n$, since $\bL$ is expected pairs at particular distance $\a n$ and there at most $n+1$ distances. By Markov's inequality, this means that w.h.p. the number of pairs of solutions at dist. $\D n$ is $poly(n) g_k^n$. Since every region has size at most $\D n$, w.h.p. the number of pairs in each region is $poly(n) g_k^n$. and if we can show that $g_k < \L_w(1/2)$ we can see that $\sf$ has at least
$$
\f{1}{poly(n)}\ \rbrac{\f{\L_w(1/2)}{g_k}}^{n/2}
$$
clusters. The rest of the argument then shows that this happens for $\a = 1/k$ and $\b = 1/2 - 5/6\sqrt{\d}$ in Thm.~\ref{thm:shatter}. We do not discuss it here because it is just careful arithmetic, the case for $k > 15$ is analytical and $15 \geq k \geq 8$ is computational. Let us however note that Thm.~\ref{thm:shatter} is quite illuminative, as $r$ approaches the threshold $2^k \log 2$, the clusters are maximally far apart, in fact their size $1/k$ decreases and they vanish as soon as we cross this threshold. Fig.~\ref{fig:phase_chart} shows this phase transition phenomenon for 3-$\SAT$.

Let us note that, using the results in this section, we can show~\cite{achlioptas2008algorithmic} that there exists a $\e_k \to 0$ s.t. the ``glassy'' phase with exponentially-many clusters exists for
$$
(1 +\e_k) \f{2^k}{k} \log 2 \leq r \leq (1-\e_k)\ 2^k \log 2.
$$

\begin{figure}
\centering
\includegraphics[width=.7\columnwidth]{phase_chart.jpg}
\caption{Phase chart for random $\ksat$}
\label{fig:phase_chart}
\end{figure}

\subsection{Frozen variables and survey propagation}
\label{ssec:frozen_variables}

% \section{1-step replica symmetry breaking}
% \label{sec:1rsb}

\section{Summary}
\label{sec:summary}


{
\small
\bibliography{writeup}
\bibliographystyle{abbrv}
}
\end{document}
