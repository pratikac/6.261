\usepackage{amsmath, amsthm, amssymb, bm, enumerate, url, setspace}
\usepackage[usenames,dvipsnames,svgnames,table]{xcolor}
\usepackage[pdftex, xetex]{graphicx}
\usepackage[font={small}]{caption}
\usepackage{float, colortbl, tabularx, multirow, subcaption, environ, wrapfig, textcomp}
\usepackage[normalem]{ulem}

\usepackage{blindtext}

\usepackage[bookmarks=true]{hyperref}
%\usepackage[hyperpageref]{backref}
\hypersetup{
colorlinks=true, linkcolor=blue, citecolor=darkgreen, filecolor=magenta, urlcolor=blue,
}
\usepackage{url, cite}

\newcommand{\pc}[2]{{\color{blue} #1}\marginpar{\tiny\noindent{\raggedright{\color{darkgreen}[PC]}\color{brown}{ #2} \par}}}
\newcommand{\todo}[1]{{\color{gray}#1}\marginpar{\tiny\noindent{\raggedright{\color{forestgreen}[MILD TODO]}}}}

%\newcommand{\myclearpage}{\clearpage}
\newcommand{\myclearpage}{}

% math macros
\newcommand{\abs}[1]{\ensuremath \left| #1 \right|}
\newcommand{\norm}[1]{\ensuremath \lVert#1\rVert}
\newcommand{\given}{\, \vert \,}
\providecommand{\cal}[1]{\ensuremath \mathcal{#1}}
\newcommand{\ag}[1]{\ensuremath \left\langle#1\right\rangle}

% shorcuts
\newcommand{\aeq}[1]{\begin{align} #1 \end{align}}
\newcommand{\aeqs}[1]{\begin{align*} #1 \end{align*}}
\newcommand{\beq}[1]{\begin{equation}#1\end{equation}}
\newcommand{\beqs}[1]{\begin{equation*}#1\end{equation*}}
\newcommand{\trm}[1]{\ensuremath \textrm{#1}}
\newcommand{\enum}[2]{\begin{enumerate}[#1]{#2}\end{enumerate}}
\newcommand{\clist}[1]{\begin{itemize}\setlength{\itemsep}{0pt}
\setlength{\parsep}{0pt}
{#1}\end{itemize}}
\newcommand{\bmat}[1]{\begin{bmatrix}#1\end{bmatrix}}
\newcommand{\mpage}[2]{\begin{center}
\begin{minipage}{#1}#2\end{minipage}\end{center}}
\newcommand{\la}{\leftarrow}
\newcommand{\ra}{\rightarrow}
\providecommand\f[2]{\ensuremath \frac{#1}{#2}}
\providecommand\rbrac[1]{\ensuremath \left(#1\right)}
\providecommand\sqbrac[1]{\ensuremath \left[#1\right]}

\newtheorem{theorem}{Theorem}
\newtheorem{proposition}[theorem]{Proposition}
\newtheorem{lemma}[theorem]{Lemma}
\newtheorem{corollary}[theorem]{Corollary}
\newtheorem{problem}[theorem]{Problem}

\theoremstyle{definition}
\newtheorem{definition}[theorem]{Definition}
\newtheorem{example}[theorem]{Example}
\newtheorem{note}[theorem]{Note}
\newtheorem{remark}[theorem]{Remark}

\newtheorem{assumption}[theorem]{Assumption}
\providecommand{\qed}{\hfill \mbox{\raggedright \rule{0.1in}{0.1in} } }

% gray proof
\let\myoldproof=\proof
\let\myoldendproof=\endproof
\renewenvironment{proof}
{\myoldproof\color{gray}}
{\myoldendproof}

% notation
\renewcommand{\P}{\trm{P}}
\newcommand{\E}{\trm{E}}
\providecommand{\ones}{\mathbb{1}}
\providecommand{\ind}{{\bf 1}}
\providecommand{\SAT}{\trm{SAT}}
\providecommand{\NAESAT}{\trm{NAE-SAT}}
\providecommand\sat{\models}
\providecommand\bigo{O}
\providecommand\smallo{o}
\providecommand\ksat{k\trm{-SAT}}
\providecommand\n[1]{\overline{#1}}
\providecommand{\oor}{\vee}
\providecommand{\aand}{\wedge}
%\providecommand{\implies}{\rightarrow}
\renewcommand{\sf}{S(F)}
\renewcommand{\sc}{S(c)}
\providecommand\s{\sigma}
\renewcommand{\t}{\tau}
\renewcommand{\a}{\alpha}
\renewcommand{\L}{\Lambda}
\newcommand{\wsf}{w(\s,F)}
\newcommand{\wsc}{w(\s,c)}
\newcommand{\wv}{w(v)}
\newcommand{\wu}{w(u)}
\newcommand{\wtf}{w(\t,F)}
\newcommand{\wtc}{w(\t,c)}
\newcommand{\puva}{\Phi_{u,v}(\a)}
